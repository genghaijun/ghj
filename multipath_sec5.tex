\section{Conclusion}\label{conclusion}
In this paper, we propose IAC, an algorithm that can efficiently 
compute loop free alternate next hops in link state networks. 
These alternates can effectively protect traffic and applications from 
network component failures, and increase the network availability. 
We carefully analyze different LFA criteria, and find the critical 
point is to compute the cost from any neighbor to each destination. 
We then push forward a further understanding of shortest paths, which are 
fundamental properties of graphs, as shown in Theorems \ref{newspt} 
$\sim$ \ref{fan}. Based on these results, and the 
ball-and-string framework proposed by Paolo Narv��ez et al. 
for incremental shortest path computation, we build IAC. 
Although IAC seems to be only a special case of iSPF with a special 
link cost update, its correctness needs non-trivial proofs. 
Our evaluation with both real and synthetic topologies shows that, 
IAC always achieves less complexity than the other algorithms, 
and provides better failure repair capability and network availability.
In the future, we will study whether tighter bounds on IAC's complexity 
can be formally established, 
since in our experiments we see a constant pattern that IAC runs faster 
than constructing a single SPT. We will also study how IAC 
can be utilized in multipath schemes for better utilization of bandwidth, 
load balancing and et al.

